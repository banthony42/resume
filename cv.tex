\documentclass[a4paper]{limecv}

\usepackage{enumitem}
\usepackage{hyperref}
\usepackage{xcolor}
\usepackage{graphicx}

% Defaults used in template design.
\usepackage[margin=\cvMargin,noheadfoot]{geometry}
 
% Define which language to use
\usepackage[fr]{optional}

% Languages settings
\opt{en}{\usepackage[english]{babel}}
\opt{en}{\cvSetLanguage{english}}

\opt{fr}{\usepackage[french]{babel}}
\opt{fr}{\cvSetLanguage{french}}

% Syntactic sugar to write content specific to a language
\newcommand{\en}[1]{\opt{en}{#1}}
\newcommand{\fr}[1]{\opt{fr}{#1}}

\newlist{myitems}{enumerate}{2}

\setlist[myitems, 1]
{label=\textbullet,
topsep=2.5pt,
parsep=1.5pt,
leftmargin=8pt,
rightmargin=0pt
}

\setlist[myitems, 2]
{label=\textbullet,
topsep=0pt,
parsep=1.5pt,
leftmargin=15pt,
rightmargin=15pt}


\newcommand{\enum}[1]{\begin{myitems}{#1}\end{myitems}}

% Overwrite default package color
% Sadly this is the only way to change color
\definecolor{cvGreen}{HTML}{26334d}         % Sidebar separators
\definecolor{cvGreenLight}{HTML}{314972}    % Sidebar background
\definecolor{cvAccent}{HTML}{FFFFFF}        % Sidebar font
\definecolor{cvDark}{HTML}{071952}          % Main titles, separators and dots
\definecolor{cvBackground}{HTML}{FFFFFF}    % Main background

% Custom colors created for this document
\definecolor{mainFontColor}{HTML}{071952}
\newcommand{\SideBarLinkColor}[1]{\color{cyan}#1\color{white}}
\newcommand{\MainLinkColor}[1]{\color{cyan}#1\color{mainFontColor}}
\newcommand{\DocumentVersionColor}[1]{\color{071952}#1\color{mainFontColor}}
\newcommand{\SideBarVSpace}{\vspace{0.5cm}}
\newcommand{\DocumentVersionSpaceOne}{\vspace{2cm}}
\newcommand{\DocumentVersionSpaceTwo}{\vspace{9cm}}

\newcommand{\cvContactSideBar}{
  \begin{cvContact}
    \cvContactAddress{Valenciennes (59)}
    \cvContactEmail{mailto:abara.pro24@gmail.com}{\SideBarLinkColor{abara.pro24@gmail.com}}
    \cvContactPhone{+33 6 31 54 80 24}
    \cvContactWebsite{https://github.com/banthony42}{\SideBarLinkColor{github.com/banthony42}}
    \cvContactLinkedin{https://www.linkedin.com/in/anthony-bara-65b2a8125/}{\SideBarLinkColor{anthonybara}}
  \end{cvContact}
}

\newcommand{\DocumentVersion}{
  \begin{center}
    \color{cvGreen}
    \href{https://github.com/banthony42/resume}{Version 0.0.1\\{\faExternalLink*} powered by LaTeX}
  \end{center}
}

\begin{document}
\begin{cvSidebar}
  \cvID{Anthony}{Bara}{assets/aba_360.jpg}{\en{Software Engineer}\fr{Développeur Logiciel}}

  \begin{cvProfile}
    \en{From working with Grafcet on old programmable logic controllers to studying on the renowed French educational platform OpenClassRooms, and finally joining 42 school.\\My path to becoming a software engineer has been an exciting one, and each new road I take now is even more so. Passionate about programming and eager to learn, over time I have learned over time that my main strength is my tenacity.}
    \fr{Des Grafcets sur de vieux automates industriels, à OpenClassRoom en passant par l'école 42.\\Le chemin pour devenir développeur fut passionnant et chaque nouvelle route que j'emprunte à présent l'est plus encore. Passionné de programmation et curieux d'apprendre,\\ le temps m'a appris que ma principale force est ma tenacité.}
  \end{cvProfile}
  \SideBarVSpace
  \cvContactSideBar

  \SideBarVSpace
  \begin{cvLanguages}
    \cvLanguage{\en{French (native)}\fr{Français (natif)}}{5}
    \cvLanguage{\en{English}\fr{Anglais}}{3}
  \end{cvLanguages}

  \SideBarVSpace
  \begin{cvInterests}[short]
    \cvInterest{\faMusic}{\en{Guitar}\fr{Guitare}}
    \cvInterest{\faGamepad}{\en{Gaming}\fr{Jeux vidéo}}
    \cvInterest{\faPen}{\en{3D modelling, pixel art}\fr{Modélisation 3D, Pixel art}}
    \cvInterest{\faTree}{\en{Fishing}\fr{Pêche}}
  \end{cvInterests}
  \DocumentVersionSpaceOne
  \DocumentVersion
\end{cvSidebar}

\begin{cvMainContent}

  \begin{cvExperience}
    \cvItem{
      \color{mainFontColor}
      \en{\textbf{Software engineer}}
      \fr{\textbf{Ingénieur logiciel sur un Système de Détection d'Intrusion réseau.}}
      \\
      \textsc{\selectfont P1 Security}, Paris. \en{Since January}\fr{Depuis Janvier} 2020\\
      \en{\enum{
          \item As part of the Network Intrusion Detection System (NIDS) development team, focusing on Telecom signaling traffic.
          \item Development, maintenance, technical debt resolution and testing across all layers of the software.
          \item \textnormal{\textit{Backend}}: C, Lua, Python, Django, Bash
          \begin {myitems}
          \item Rework and improvement of the asynchronous REST API client, with stong configuration constraints.
          \item Improvement of the packet history system
          \item Rewriting and improvement of the Elasticsearch instance manager (Mappings, Templates, Lifecycles)
          \item Added support for OpenIDConnect
          \end {myitems}
          \item \textnormal{\textit{Frontend}}: CSS, JS, Kibana
          \begin {myitems}
          \item Creation of Kibana visualizations and dashboards.
          \item Integrations of Kibana views into the software's web interface.
          \end {myitems}
          \item \textnormal{\textit{Database}}: Redis, Elasticsearch
          \begin {myitems}
          \item Production-ready tool for handling automatic database migration after Elasticsearch upgrade.
          \item Resolution of data format changes, without loss (10 to 40 million documents).
          \end {myitems}
          \item \textnormal{\textit{Continuous Integration}}: Jenkins, Python, bash
          \begin {myitems}
          \item Jenkins devops tasks (one master node and five agents).
          \item VMs maintenance, test pipeline updates, unit tests, functional tests, end-to-end tests.
          \end {myitems}
        }
      }
      \fr{\enum{
          \item Au sein de l'équipe de développement d'un IDS pour les réseaux Telecom, sur le traffic de signalisation.
          \item Développement, maintenance, résolution de la dette technique et test sur toutes les couches du logiciel.
          \item \textnormal{\textit{Backend}}: C, Lua, Python, Django, Bash\\
          \begin {myitems}
          \item Réécriture et amélioration d'un client API REST asynchrone, avec de fortes contraintes de configuration.
          \item Amélioration du système d'historisation des paquets.
          \item Réécriture et amélioration d'un gestionnaire d'instance Elasticsearch. (Mappings, Templates, Lifecycles)
          \item Ajout de la prise en charge d'OpenIDConnect.
          \end {myitems}
          \item \textnormal{\textit{Frontend}}: CSS, JS, Kibana\\
          \begin {myitems}
          \item Création de visualisations et dashboard dans Kibana.
          \item Intégration de vues Kibana dans l'interface web du logiciel.
          \end {myitems}
          \item \textnormal{\textit{Base de données}}: Redis, Elasticsearch
          \begin {myitems}
          \item Outil de migration automatique de la base de données, en production, suite à la mise à jour d'Elasticsearch.\\
          \item Résolution des changements de format de données, sans pertes (10 à 40 millions de documents).
          \end {myitems}
          \item \textnormal{\textit{Tests d'intégration continue}}: Jenkins, Python, Bash\\
          \begin {myitems}
          \item Devops sur une CI Jenkins (un master et cinq agents).\\
          \item Maintenance et développement: pipeline de tests, tests unitaires, tests fonctionnels et de bout en bout.
          \end {myitems}
        }}
    }
    \cvItem{
      \color{mainFontColor}
      \en{\textbf{SDK software developer for a robot.}}
      \fr{\textbf{Développeur logiciel SDK pour robot.}}
      \\
      \textsc{\selectfont Blue Frog Robotics}, Paris. \en{May}\fr{Mai} 2019 -- \en{October}\fr{Octobre} 2019\\
      \en{\enum{
          \item Development of new features for the robot, focusing on providing a good public interface for the SDK users (C\#/Unity).
          \item Maintenance of the existing code base.
          \item Code reviews and agile methodology.
        }}
      \fr{\enum{
          \item Développement de nouvelles fonctionnalités avec une attention particulière pour fournir une interface de qualité pour les utilisateurs du SDK (C\#/Unity).
          \item Réécriture et maintenance du code existant.
          \item Revue de code et méthode agile.
        }}
    }
    \cvItem{
      \color{mainFontColor}
      \en{\textbf{Robot application developer.}}
      \fr{\textbf{Développeur d'applications pour robot.}}
      \\
      \textsc{\selectfont Blue Frog Robotics}, Paris. \en{September}\fr{Septembre} 2018 -- \en{March}\fr{Mars} 2019\\
      \en{\enum{
          \item Application development using internal SDK (C\#/Unity).
          \item Development of an app to easily run non-regression automated tests.
          \item Rework and maintenance of existing apps.
          \item Code reviews and agile methodology.
        }}
      \fr{\enum{
          \item Développement d'applications basé sur le SDK interne (C\#/Unity).
          \item Développement de tests non régression automatisés.
          \item Réécriture et maintenance des applications existantes.
          \item Revue de code et méthode agile.
        }}
    }
    \cvItem{
      \color{mainFontColor}
      \en{\textbf{Embedded development, mechanical and electronic design}}
      \fr{\textbf{Développement embarqué, conception mécanique et électronique}}\\
      \textsc{Mainbot} Paris. \en{November}\fr{Novembre} 2016 -- \en{April}\fr{Avril} 2017\\
      \en{\enum{
          \item Development of the first Winky robot prototype.
          \item Mechanical design for 3D printing, using Solidworks.
          \item Printed circuit board design using Kicad.
          \item Development on Arduino and Raspberry Pi. (C/C++)
        }}
      \fr{\enum{
          \item Développement du premier prototype du robot Winky
          \item Conception mécanique sous Solidworks pour impression 3D.
          \item Conception de circuits imprimés sous Kicad.
          \item Programmation sur Arduino et Raspberry Pi. (C/C++)
        }
      }}

    \cvItem{
      \color{mainFontColor}
      \en{\textbf{Industrial designer}}
      \fr{\textbf{Dessinateur-Projeteur industriel}}\\
      \textsc{Boubiela Moret} Saint-Quentin. \en{May}\fr{Mai} 2013 -- \en{November}\fr{Novembre} 2015\\
      \en{\enum{
          \item Design of loading equipment according to specifications.
          \item Creation of 3D models and fabrication plans using Solidworks.
          \item Monitoring of part manufacturing and machine assembly.
        }}
      \fr{\enum{
          \item Conception de machines de manutention selon cahier des charges.
          \item Création de modèles 3D et plans de fabrication sur Solidworks.
          \item Suivi de fabrication et du montage des pièces.
        }}};
  \end{cvExperience}
\end{cvMainContent}

\newpage

\begin{cvSidebar}
  \cvID{Anthony}{Bara}{assets/aba_360.jpg}{\en{Software Engineer}\fr{Développeur Logiciel}}

  \cvContactSideBar

  \SideBarVSpace
  \begin{cvProjects}
    \cvProject[link=https://github.com/banthony42/rpg]{\SideBarLinkColor{\texttt{Rusted Adventures}}}{
      \en{\enum{
          \color{white}
          \item Enjoy learning Rust while developing a multiplayer RPG.
          \item Status: In progress
          \item Initiative: Personal project
        }}
      \fr{\enum{
          \color{white}
          \item S'amuser à apprendre Rust en développant un RPG multijoueur.
          \item Statut: En cours
          \item Initiative: Projet Personnel
        }}
    }
    \cvProject[link=https://github.com/banthony42/xv]{\SideBarLinkColor{\texttt{XV}}}{
      \en{\enum{
          \color{white}
          \item Animated industrial tasks editor and viewer.
          \item Status: Finished
          \item Initiative: 42 School
        }}
      \fr{\enum{
          \color{white}
          \item Visualisation et création de tâches industrielles animées.
          \item Statut: Terminé
          \item Initiative: Ecole 42
        }}
      \DocumentVersionSpaceTwo
      \DocumentVersion
    }
  \end{cvProjects}
\end{cvSidebar}

\begin{cvMainContent}

  \begin{cvEducation}
    \cvItem{
      \color{mainFontColor}
      \en{\textbf{Rust Learning}}
      \fr{\textbf{Apprentissage du langage Rust}}\\
      \en{Autodidact}\fr{Autodidacte}. \en{Since February}\fr{Depuis Février} 2024\\
      \en {\enum {
          \item Started learning by reading documentation (Rust Book).
          \item Then practiced with basic exercices, data import with JSON and basic client-server chat implementation.
          \item Creation of \MainLinkColor{\href{https://github.com/banthony42/rpg}{Rusted Adventures}}, a multiplayer RPG project:
          \begin {myitems}
          \item The only ambition is to learn Rust and enjoy trying to develop a multiplayer game.
          \item Graphic client using 'piston' crate.
          \item Client-server communicate using \MainLinkColor{\href{https://grpc.io/}{gRPC }} with 'tonic'.
          \item Server-database communicate using an ORM, with 'diesel'.
          \item Asynchronous tasks for requests, using 'tokyo'.
          \item CLI tool to manage players accounts, using 'clap'.
          \item Tilemaps are imported as JSON to practice with 'serde'.
          \item Game maps drawn in pixel art using \MainLinkColor{\href{https://www.aseprite.org/}{Aseprite}}.
          \end {myitems}
        }}
      \fr{\enum{
          \item Début d'apprentissage par lecture de documentation (Rust Book).
          \item Mise en pratique avec des excercices basiques, import de données au format JSON, implémentation d'un chat basique (client-serveur).
          \item Création de \MainLinkColor{\href{https://github.com/banthony42/rpg}{Rusted Adventures}}, un RPG multijoueur:
          \begin {myitems}
          \item La seule ambition est d'apprendre le Rust, tout en prenant plaisir à coder un jeu multijoueur.
          \item Client graphique basé sur le crate 'piston'.
          \item Client et serveur communiquent en \MainLinkColor{\href{https://grpc.io/}{gRPC }} avec 'tonic'.
          \item Serveur et base de données communiquent au moyen d'un ORM, avec 'diesel'.
          \item Tâches asynchrones pour les requêtes avec 'tokyo'.
          \item Outil CLI pour gérer les comptes de joueurs avec 'clap'.
          \item Import des tilemaps en JSON pour pratiquer avec 'serde'.
          \item Cartes du jeu dessinées en pixel art avec \MainLinkColor{\href{https://www.aseprite.org/}{Aseprite}}.
          \end {myitems}
        }}
    }

    \cvItem{
      \color{mainFontColor}
      \en{\textbf{Certificate of Architect in digital technologies}}
      \fr{\textbf{Certificat architecte du numérique}}\\
      \begin{flushright}
        \vspace{-1cm}
        \includegraphics[keepaspectratio=true,height=2.5cm]{assets/42-cursus-computer-architect-level-21.1.png}
        \hspace{0.2cm}
        \vspace{-1.5cm}
      \end{flushright}
      \en{42 School}\fr{Ecole 42} Paris. \en{November}\fr{Novembre} 2015 -- \en{January}\fr{Janvier} 2020 \\
      \en{
        \enum{
          \item Learning through project implementation.
          \item Peer code reviews, and reading documentation.
          \item Intensive C practice by rewriting well-known programs such as malloc, client-server ftp, basics of Wolfenstein 3D.\\
          \item Nibbler, a snake game with several graphic interfaces by using dynamic graphic library switching at runtime (C++).
          \item \MainLinkColor{\href{https://github.com/banthony42/xv}{XV}}, an animated industrial task editor and viewer. (C\# - Unity).
        }}
      \fr {
        \enum{
          \item Apprentissage par la réalisation de projets.
          \item Revue de code par les pairs et lecture de documentation.
          \item Pratique intensive du C par la réécriture de programmes bien connu comme malloc, un client-serveur ftp, un Wolfenstein 3D basique.
          \item Nibbler, un jeu snake avec differentes interfaces graphiques. Chargement de bibliothèques dynamiques au runtime (C++).
          \item \MainLinkColor{\href{https://github.com/banthony42/xv}{XV}}, permet de visualiser et créer des tâches industrielles animées. (C\# - Unity).
        }
      }
    }

    \cvItem{
      \color{mainFontColor}
      \en{\textbf{Bachelor's Degree in Computer-Aided Design engineering}}
      \fr{\textbf{Licence Professionnelle Ingénierie de la Conception Informatisée}}\\
      UPJV Saint-Quentin. \en{September}\fr{Septembre} 2012 -- \en{May}\fr{Mai} 2013\\
      \en{\enum{
          \item Mastery of CAD tools.
          \item Mechanical engineering studies and industrial project management.
        }
      }
      \fr{\enum{
          \item Maîtrise des outils de dessin assisté par ordinateurs (DAO).
          \item Ingénierie mécanique et gestion de projets industriels.
        }
      }
    }

    \cvItem{
      \color{mainFontColor}
      \en{\textbf{BTEC Industrial Mecanic and Automation}}
      \fr{\textbf{BTS Mécanique Automatisme Industriel}}\\
      \en{Condorcet High school}\fr{Lycée Condorcet} Saint-Quentin. \en{September}\fr{Septembre} 2010 -- \en{June}\fr{Juin} 2012\\
      \en{\enum{
          \item Design and maintenance of industrial machines and automated systems.
        }
      }
      \fr{\enum{
          \item Conception et maintenance de systèmes automatisés industriels.
        }
      }}
  \end{cvEducation}

  \begin{cvSkills}
    \cvSkillTwo{4}{C}{4}{Linux}
    \cvSkillTwo{4}{Lua}{4}{Elasticsearch}
    \cvSkillTwo{4}{Bash}{4}{Kibana}
    \cvSkillTwo{3}{Python \& Django}{4}{Redis}
    \cvSkillTwo{3}{VirtualBox / Vagrant}{2}{VMWare/ESXi}
    \cvSkillTwo{3}{Rust - \en{\textbf{Active learning}}\fr{\textbf{Apprentissage actif}}}{2}{Javascript}
    \cvSkillTwo{2}{HTML/CSS}{1}{Docker}
    \cvSkillTwo{1}{C\#}{1}{C++}
    \cvSkillOne{1}{PostgreSQL}
  \end{cvSkills}
\end{cvMainContent}

% \clearpage

% \begin{cvCoverLetter}

%   \cvBeneficiary{%
%     name=Jane Smith,
%     position=Position,
%     company=Company,
%     address line 1=Address line 1,
%     address line 2=Address line 2}

%   Dear Miss.\ Smith

%   \vspace{\baselineskip}

%   Lorem ipsum dolor sit amet, consectetur adipiscing elit. Morbi dictum cursus sapien, id eleifend mi pellentesque id. Etiam lobortis eu odio a sodales. Phasellus ut dolor feugiat, lacinia lectus in, blandit metus.  Fusce lacinia dolor et metus gravida pulvinar sit amet et ex. Suspendisse vestibulum, leo malesuada molestie maximus, sem risus ornare elit, vitae sodales felis elit in ipsum.
%   Lorem ipsum dolor sit amet, consectetur adipiscing elit. Morbi dictum cursus sapien, id eleifend mi pellentesque id. Etiam lobortis eu odio a sodales. Phasellus ut dolor feugiat, lacinia lectus in, blandit metus.  Fusce lacinia dolor et metus gravida pulvinar sit amet et ex. Suspendisse vestibulum, leo malesuada molestie maximus, sem risus ornare elit, vitae sodales felis elit in ipsum.
%   Lorem ipsum dolor sit amet, consectetur adipiscing elit. Morbi dictum cursus sapien, id eleifend mi pellentesque id. Etiam lobortis eu odio a sodales. Phasellus ut dolor feugiat, lacinia lectus in, blandit metus.  Fusce lacinia dolor et metus gravida pulvinar sit amet et ex. Suspendisse vestibulum, leo malesuada molestie maximus, sem risus ornare elit, vitae sodales felis elit in ipsum.

%   \vspace{\cvMargin}

%   \cvFullName

% \end{cvCoverLetter}

\end{document}