\documentclass[a4paper]{limecv}

\usepackage{enumitem}
\usepackage{hyperref}
\usepackage{xcolor}

% Defaults used in template design.
\usepackage[margin=\cvMargin,noheadfoot]{geometry}
 
% Define which language to use
\usepackage[en]{optional}

% Languages settings
\opt{en}{\usepackage[english]{babel}}
\opt{en}{\cvSetLanguage{english}}

\opt{fr}{\usepackage[french]{babel}}
\opt{fr}{\cvSetLanguage{french}}

% Syntactic sugar to write content specific to a language
\newcommand{\en}[1]{\opt{en}{#1}}
\newcommand{\fr}[1]{\opt{fr}{#1}}

\newlist{myitems}{enumerate}{2}

\setlist[myitems, 1]
{label=\textbullet,
topsep=2.5pt,
parsep=1.5pt,
leftmargin=8pt,
rightmargin=0pt
}

\setlist[myitems, 2]
{label=\textbullet,
topsep=0pt,
parsep=1.5pt,
leftmargin=15pt,
rightmargin=15pt}


\newcommand{\enum}[1]{\begin{myitems}{#1}\end{myitems}}

% Overwrite default package color
% Sadly this is the only way to change color
\definecolor{cvGreen}{HTML}{26334d}         % Sidebar separators
\definecolor{cvGreenLight}{HTML}{314972}    % Sidebar background
\definecolor{cvAccent}{HTML}{FFFFFF}        % Sidebar font
\definecolor{cvDark}{HTML}{071952}          % Main titles, separators and dots
\definecolor{cvBackground}{HTML}{FFFFFF}    % Main background

% Custom colors created for this document
\definecolor{mainFontColor}{HTML}{071952}
\newcommand{\SideBarLinkColor}[1]{\color{cyan}#1\color{white}}
\newcommand{\MainLinkColor}[1]{\color{cyan}#1\color{mainFontColor}}

\newcommand{\cvContactSideBar}{
  \begin{cvContact}
    \cvContactAddress{Valenciennes (59)}
    \cvContactEmail{mailto:abara.pro24@gmail.com}{\SideBarLinkColor{abara.pro24@gmail.com}}
    \cvContactPhone{+33 6 31 54 80 24}
    \cvContactWebsite{https://github.com/banthony42}{\SideBarLinkColor{github.com/banthony42}}
    \cvContactLinkedin{https://www.linkedin.com/in/anthony-bara-65b2a8125/}{\SideBarLinkColor{anthonybara}}
  \end{cvContact}
}

\begin{document}

\begin{cvSidebar}

  \cvID{Anthony}{Bara}{assets/aba_360.jpg}{\en{Software Engineer}\fr{Développeur Logiciel}}

  \begin{cvProfile}
    \en{From Grafcet on old programmable logic controller, to the famous french educational platform OpenClassRoom, to finally land at 42 school.\\The path to becoming a developer was exciting and each new road I take now is even more so. Passionate about programming and curious to learn. My recent professional years have taught me that my main strength is my tenacity.}
    \fr{Des Grafcets sur de vieux automates industriels, à OpenClassRoom en passant par l'école 42.\\Le chemin pour devenir développeur fut passionnant et chaque nouvelle route que j'emprunte à présent l'est plus encore. Passionné de programmation et curieux d'apprendre.\\Mes récentes années professionnelles m'ont enseignées que ma principale force est ma tenacité.}
  \end{cvProfile}

  \cvContactSideBar

  \begin{cvLanguages}
    \cvLanguage{\en{French (native)}\fr{Français (natif)}}{5}
    \cvLanguage{\en{English}\fr{Anglais}}{3}
  \end{cvLanguages}

  \begin{cvInterests}[short]
    \cvInterest{\faMusic}{\en{Guitar}\fr{Guitare}}
    \cvInterest{\faGamepad}{\en{Gaming}\fr{Jeux vidéo}}
    \cvInterest{\faPen}{\en{3D modelling, pixel art}\fr{Modélisation 3D, Pixel art}}
    \cvInterest{\faTree}{\en{Fishing}\fr{Pêche}}
  \end{cvInterests}
\end{cvSidebar}

\begin{cvMainContent}

  \begin{cvExperience}
    \cvItem{
      \color{mainFontColor}
      \en{\textbf{Software engineer}}
      \fr{\textbf{Ingénieur logiciel sur un Système de Détection d'Intrusion réseau.}}
      \\
      \textsc{\selectfont P1 Security}, Paris. \en{since January}\fr{Depuis Janvier} 2020\\
      \en{\enum{
          \item Within Network Intrusion Detection System development team, for Telecom signaling traffic.
          \item Development, maintainance and test all the staks of the software.
          \item Technical debt resolution.
          \item \textnormal{\textit{Backend}}: C, Lua, Python, Django, Bash
          \begin {myitems}
          \item Rework and improvement of the asynchronous API REST client, keeping it highly configurable.
          \item Improvement of the packet history system
          \item Rewrite and improvement of the Elasticsearch instance manager (Mappings, Templates, Lifecycles)
          \item Added support for OpenIDConnect
          \end {myitems}
          \item \textnormal{\textit{Frontend}}: CSS, JS, Kibana
          \begin {myitems}
          \item Creation of Kibana visualizations and dashboards.
          \item Kibana pages integrations into the software's web interface.
          \end {myitems}
          \item \textnormal{\textit{Database}}: Redis, Elasticsearch
          \begin {myitems}
          \item Production ready tool, to handle automatic database migration after Elasticsearch upgrade. (5.6 -> 7.10)
          \item Resolutions of data format changes, without losses (10 to 40 million documents).
          \end {myitems}
          \item \textnormal{\textit{Continuous integration}}: Jenkins, Python, bash
          \begin {myitems}
          \item Devops on a Jenkins CI composed of a master node and five agents.
          \item Maintenance and development: test pipeline, unit tests, functional / end-to-end tests.
          \end {myitems}
        }
      }
      \fr{\enum{
          \item Au sein de l'équipe de développement d'un IDS pour les réseaux Telecom, sur le traffic de signalisation.
          \item Développement, maintenance et test sur toutes les couches du logiciel.
          \item Résolutions de la dette technique.
          \item \textnormal{\textit{Backend}}: C, Lua, Python, Django, Bash\\
          \begin {myitems}
          \item Réécriture et amélioration d'un client API REST asynchrone, avec de fortes contraintes de configuration.
          \item Amélioration du système d'historisation des paquets
          \item Réécriture et amélioration d'un gestionnaire d'instance Elasticsearch. (Mappings, Templates, Lifecycles)
          \item Ajout de la prise en charge d'OpenIDConnect.
          \end {myitems}
          \item \textnormal{\textit{Frontend}}: CSS, JS, Kibana\\
          \begin {myitems}
          \item Création de visualisations et dashboard dans Kibana.
          \item Intégration de vues Kibana dans l'interface web du logiciel.
          \end {myitems}
          \item \textnormal{\textit{Base de données}}: Redis, Elasticsearch
          \begin {myitems}
          \item Outil de migration automatique de la base de données, en production, suite à la mise à jour d'Elasticsearch\\(5.6 -> 7.10).
          \item Résolutions des changements de format de données, sans pertes (10 à 40 millions de documents).
          \end {myitems}
          \item \textnormal{\textit{Tests d'intégration continue}}: Jenkins, Python, Bash\\
          \begin {myitems}
          \item Devops sur une CI Jenkins composé d'un noeud master et de cinq agents.\\
          \item Maintenance et développement: pipeline de tests, tests unitaires, fonctionnels et de bout en bout.
          \end {myitems}
        }}
    }
    \cvItem{
      \color{mainFontColor}
      \en{\textbf{SDK software developer for a robot.}}
      \fr{\textbf{Développeur logiciel SDK pour robot.}}
      \\
      \textsc{\selectfont Blue Frog Robotics}, Paris. \en{May}\fr{Mai} 2019 -- \en{October}\fr{Octobre} 2019\\
      \en{\enum{
          \item New feature development for the robot, keeping in mind to supply good public interface for the SDK users (C\#/Unity).
          \item Maintenance of the existing code base.
          \item Code review and agile method.
        }}
      \fr{\enum{
          \item Développement de nouvelles fonctionnalités avec une attention particulière pour fournir une interface de qualité pour les utilisateurs du SDK (C\#/Unity).
          \item Réécriture et maintenance du code existant.
          \item Revue de code et méthode agile.
        }}
    }
    \cvItem{
      \color{mainFontColor}
      \en{\textbf{Robot application developer.}}
      \fr{\textbf{Développeur d'applications pour robot.}}
      \\
      \textsc{\selectfont Blue Frog Robotics}, Paris. \en{September}\fr{Septembre} 2018 -- \en{March}\fr{Mars} 2019\\
      \en{\enum{
          \item Applications development using internal SDK (C\#/Unity).
          \item Development of an app to easily run non-regression automated tests.
          \item Rework and maintenance of existing apps.
          \item Code review and agile method.
        }}
      \fr{\enum{
          \item Développement d'applications basé sur le SDK interne (C\#/Unity).
          \item Développement de tests automatisés de non régression.
          \item Réécriture et maintenance des applications existantes.
          \item Revue de code et méthode agile.
        }}
    }
    \cvItem{
      \color{mainFontColor}
      \en{\textbf{Embedded development and mechanic, electronic conception}}
      \fr{\textbf{Développement embarqué, conception mécanique et électronique}}\\
      \textsc{Mainbot} Paris. \en{November}\fr{Novembre} 2016 -- \en{April}\fr{Avril} 2017\\
      \en{\enum{
          \item Winky robot first prototype development
          \item Mechanical conception for 3D printing, using Solidworks.
          \item Printed circuit board conception using Kicad.
          \item Development on Arduino and Raspberry Pi. (C/C++)
        }}
      \fr{\enum{
          \item Développement du premier prototype du robot Winky
          \item Conception mécanique sous Solidworks pour impression 3D.
          \item Conception de circuits imprimés sous Kicad.
          \item Programmation sur Arduino et Raspberry Pi. (C/C++)
        }
      }}

    \cvItem{
      \color{mainFontColor}
      \en{\textbf{Industrial designer}}
      \fr{\textbf{Dessinateur-Projeteur industriel}}\\
      \textsc{Boubiela Moret} Saint-Quentin. \en{May}\fr{Mai} 2013 -- \en{November}\fr{Novembre} 2015\\
      \en{\enum{
          \item Design of loading equipments according to specifications.
          \item Creation of 3D models and fabrication plans on Solidworks.
          \item Monitoring part manufacturing and machine assembly.
        }}
      \fr{\enum{
          \item Conception de machines de manutention selon cahier des charges.
          \item Création de modèles 3D et plans de fabrication sur Solidworks.
          \item Suivi de fabrication et du montage des pièces.
        }}};
  \end{cvExperience}
\end{cvMainContent}

\newpage

\begin{cvSidebar}
  \cvID{Anthony}{Bara}{assets/aba_360.jpg}{\en{Software Engineer}\fr{Développeur Logiciel}}

  \cvContactSideBar

  \begin{cvProjects}
    \cvProject[link=https://github.com/banthony42/rpg]{\SideBarLinkColor{\texttt{Tribute to Dofus}}}{
      \en{\enum{
          \color{white}
          \item Enjoy learning Rust trying to develop a connected RPG.
          \item Status: In progress
          \item Initiative: personal
        }}
      \fr{\enum{
          \color{white}
          \item S'amuser à apprendre Rust en développant un RPG connecté.
          \item Statut: En cours
          \item Initiative: personnel
        }}
    }
    \cvProject[link=https://github.com/banthony42/xv]{\SideBarLinkColor{\texttt{XV}}}{
      \en{\enum{
          \color{white}
          \item Animated industrial tasks viewer/editor.
          \item Status: Finished
          \item Initiative: 42 school
        }}
      \fr{\enum{
          \color{white}
          \item Visualisation et création de tâches industrielles animée.
          \item Statut: Terminé
          \item Initiative: Ecole 42
        }}
    }
  \end{cvProjects}
\end{cvSidebar}

\begin{cvMainContent}

  \begin{cvEducation}
    \cvItem{
      \color{mainFontColor}
      \en{\textbf{Rust Learning}}
      \fr{\textbf{Apprentissage du language Rust}}\\
      \en{Autodidact}\fr{Autodidacte}. \en{Since February}\fr{Depuis Février} 2024\\
      \en {\enum {
          \item Start learning by reading documentation (Rust Book).
          \item Then practice with exercices, data import with JSON and basic client-server chat implementation.
          \item Creation of \MainLinkColor{\href{https://github.com/banthony42/rpg}{Tribute to Dofus}}, the early days of a connected RPG project:
          \begin {myitems}
          \item The only ambition is to learn Rust and enjoy trying to develop a connected RPG.
          \item Graphic client using 'piston' crate.
          \item Client-server communicate using \MainLinkColor{\href{https://grpc.io/}{GRPC}} with 'tonic'.
          \item Server-database communicate using an ORM, with 'diesel'.
          \item Asynchronous tasks for requests, using 'tokyo'.
          \item CLI tool to manage players accounts, using 'clap'.
          \item Tilemaps are imported as JSON to practice with 'serde'.
          \item Game maps drawn in pixel art using \MainLinkColor{\href{https://www.aseprite.org/}{Aseprite}}.
          \end {myitems}
        }}
      \fr{\enum{
          \item Début d'apprentissage par lecture de documentations (Rust Book).
          \item Mise en pratique avec des excercices simple, import de données au format JSON, implementation d'un chat basique (client-serveur).
          \item Création de \MainLinkColor{\href{https://github.com/banthony42/rpg}{Tribute to Dofus}}, les balbutiments d'un RPG connecté:
          \begin {myitems}
          \item La seule ambition est d'apprendre le Rust, tout en prenant plaisir à coder un RPG connecté.
          \item Client graphique basé sur le crate 'piston'.
          \item Client et serveur communiquent en GRPC, avec 'tonic'.
          \item Serveur et base de données communiquent au moyen d'un ORM, avec 'diesel'.
          \item Tâches asynchrone pour les requêtes avec 'tokyo'.
          \item Outil CLI pour gérer les comptes de joueurs avec 'clap'.
          \item Import des tilemaps au format JSON pour pratiquer avec 'serde'.
          \item Cartes du jeu dessinées en pixel art avec Aseprite.
          \end {myitems}
        }}
    }

    \cvItem{
      \color{mainFontColor}
      \en{\textbf{Numeric architect diploma}}
      \fr{\textbf{Diplôme d'architecte du numérique}}\\
      \en{42}\fr{Ecole 42} Paris. \en{November}\fr{Novembre} 2015 -- \en{January}\fr{Janvier} 2020\\
      \en{
        \enum{
          \item Learning through projects implementation.
          \item Peer code review, and reading documentations.
          \item Intensive C practice rewriting well-known programm such as malloc, client-server ftp, basics of Wolfenstein 3D.\\
          \item Then object-oriented programming :
          \begin {myitems}
          \item Nibbler, a snake game with dynamic graphic library switch at runtime (C++).
          \item XV, an animated industrial tasks editor and viewer. Element animations, timeline implementation to handle the whole models animations, file manager to save the tasks simulation or to load an existing one, 3D object import at runtime (C\# - Unity).
          \end {myitems}
        }}
      \fr {
        \enum{
          \item Apprentissage du développement informatique par réalisation de projets.
          \item Revue de code par les pairs et lecture de documentation.
          \item Pratique intensive du C par la réécriture de programme bien connu comme malloc, un client-serveur ftp, un Wolfenstein 3D basique.
          \item Puis de la programmation orientée objet :
          \begin {myitems}
          \item Nibbler, un jeu snake avec chargement de lib dynamique au runtime (C++).
          \item XV, permet de visualiser et créer des tâches industrielles animée. Animation d'éléments entre eux, implémentation d'une timeline, d'un gestionnaire de fichier pour enregistrer la simulation ou en charger une existante, import d'objet 3D au runtime (C\# - Unity).
          \end {myitems}
        }
      }
    }

    \cvItem{
      \color{mainFontColor}
      \en{\textbf{Bachelor of Science in informatic conception engineering}}
      \fr{\textbf{Licence Professionnelle Ingénierie de la Conception Informatisée}}\\
      UPJV Saint-Quentin. \en{September}\fr{Septembre} 2012 -- \en{May}\fr{Mai} 2013\\
      \en{\enum{
          \item Mastery of CAD tools.
          \item Mechanical engineering studies and industrial project management.
        }
      }
      \fr{\enum{
          \item Maîtrise des outils de dessin assisté par ordinateurs (DAO).
          \item Ingénierie mécanique et gestion de projets industriels.
        }
      }
    }

    \cvItem{
      \color{mainFontColor}
      \en{\textbf{BTEC Industrial Mecanic and Automation}}
      \fr{\textbf{BTS Mécanique Automatisme Industriel}}\\
      \en{High school}\fr{Lycée} Condorcet Saint-Quentin. \en{September}\fr{Septembre} 2010 -- \en{June}\fr{Juin} 2012\\
      \en{\enum{
          \item Conception and maintenance of industrial machines and automated systems.
        }
      }
      \fr{\enum{
          \item Conception et maintenance de systèmes automatisés industriels.
        }
      }}
  \end{cvEducation}

  \begin{cvSkills}
    \cvSkillTwo{4}{C}{4}{Linux}
    \cvSkillTwo{4}{Lua}{4}{Elasticsearch}
    \cvSkillTwo{4}{Bash}{4}{Kibana}
    \cvSkillTwo{3}{Python \& Django}{4}{Redis}
    \cvSkillTwo{3}{VirtualBox / Vagrant}{2}{VMWare/ESXi}
    \cvSkillTwo{2}{HTML/CSS}{2}{Javascript}
    \cvSkillTwo{2}{C\#}{1}{C++}
    \cvSkillTwo{2}{Rust - \en{\textbf{Active learning}}\fr{\textbf{Apprentissage actif}}}{1}{Docker}
  \end{cvSkills}

\end{cvMainContent}

% \clearpage

% \begin{cvCoverLetter}

%   \cvBeneficiary{%
%     name=Jane Smith,
%     position=Position,
%     company=Company,
%     address line 1=Address line 1,
%     address line 2=Address line 2}

%   Dear Miss.\ Smith

%   \vspace{\baselineskip}

%   Lorem ipsum dolor sit amet, consectetur adipiscing elit. Morbi dictum cursus sapien, id eleifend mi pellentesque id. Etiam lobortis eu odio a sodales. Phasellus ut dolor feugiat, lacinia lectus in, blandit metus.  Fusce lacinia dolor et metus gravida pulvinar sit amet et ex. Suspendisse vestibulum, leo malesuada molestie maximus, sem risus ornare elit, vitae sodales felis elit in ipsum.
%   Lorem ipsum dolor sit amet, consectetur adipiscing elit. Morbi dictum cursus sapien, id eleifend mi pellentesque id. Etiam lobortis eu odio a sodales. Phasellus ut dolor feugiat, lacinia lectus in, blandit metus.  Fusce lacinia dolor et metus gravida pulvinar sit amet et ex. Suspendisse vestibulum, leo malesuada molestie maximus, sem risus ornare elit, vitae sodales felis elit in ipsum.
%   Lorem ipsum dolor sit amet, consectetur adipiscing elit. Morbi dictum cursus sapien, id eleifend mi pellentesque id. Etiam lobortis eu odio a sodales. Phasellus ut dolor feugiat, lacinia lectus in, blandit metus.  Fusce lacinia dolor et metus gravida pulvinar sit amet et ex. Suspendisse vestibulum, leo malesuada molestie maximus, sem risus ornare elit, vitae sodales felis elit in ipsum.

%   \vspace{\cvMargin}

%   \cvFullName

% \end{cvCoverLetter}

\end{document}